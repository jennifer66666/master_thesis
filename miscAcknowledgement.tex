Here I would like to appreciate the advice and seminars arrangement from Prof. Ishikawa, without whose 
existence I will just grigri around. And I would also express my thanks to Horiuchi-san and Tan-san in 
Ishikawa lab, who always answer my questions in terms of server and GPU usage. By the way the manual on 
lab's wiki about Dockers really helps me out. It saves me a bunch of time setting the environment that different 
repos ask for. Finally I would thank to the company I have interned for, named Pixelshift. It taught me to 
use tools such as Git and VScode to control my version and debug. Also my coding ability is greatly trained 
here. Some habits such as keeping a working diary and always arrange my files well are from there.
All those mentioned above seems trifles and just tools in development. Techniques surely change. But for me actually engineering was a 
problem. With ideas I can not test and experiment by myself. And I used to be afraid of coding because it always 
gives error that I can spend a whole day to deal with. I did work a lot and had a hard time during my fisrt intern period. Now though 
I can not say that I'm good at it, at least I will not be afraid of it. And it allows me to test and play around with 
some ideas. I believe it is true for other things in my life. If I'm afraid of something that I have to deal with, then
I need to spend more time on it, to overcome the problem in mind.  

\begin{flushright}
\par
\noindent
\today

\par
\noindent
Wang Yiwen
\end{flushright}
%\clearpage\thispagestyle{empty}